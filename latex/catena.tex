\documentclass{tufte-handout}
\title{Jason Catena}
\date{}
%\date{Functional r\'{e}sum\'{e} and portfolio}
\begin{document}
\maketitle

\marginnote{jason.catena@gmail.com

\noindent \url{http://twitter.com/catenate}

\noindent cell +1 847 344 5976	

\noindent 740 Bayberry Drive, Bartlett, IL 60103}

\begin{abstract}
To produce better code more quickly, I strive consistently to exercise
craftsmanship principles, and better tool and automate my process and
those around me.
\end{abstract}

\marginnote{Motorola work history in iDEN, WiMax, and LTE Trial.

\noindent \url{http://www.linkedin.com/in/jasoncatena}}

\subsection{Design process and architect tool set}

Designed and guided evolution of two build systems, for several
telecommunications products in three major systems (iDEN, WiMAX, LTE),
over 6.5 and 3.5 years of development.

Ported serial build system to parallel make with ElectricAccelerator by
Electric Cloud from 2007 to 2009.

Promoted and implemented daily build and continuous integration
practices since 1999.

Collecting\marginnote{\url{https://dl.getdropbox.com/u/502901/naurtf.pdf}}
academic and web discussion explicitly on programming as theory
building.


\subsection{Code and debug}

Wrote 90\% of all code for each of the two build systems from 1999 to
2009.  Root-caused all significant problems and implemented all
architectural updates and extensions.  Continue to review changes.

Added lex to refactor a special-purpose tool in C-and-expect into a
script-driven automation utility, to send commands from WindRiver Linux
platform to WindRiver VxWorks and FPGA MMI interfaces.

Wrote\marginnote{\url{http://github.com/catenate/tictestrue}} in Ruby a
tic-tac-toe game which does not lose, and finds forks to force wins when
it can.  Used the test-driven style to ensure that it meets its
requirements and to allow more assured refactoring.

Rewriting in Java a Perl program I expanded to find signatures of known
defects in logs captured from WiMax base stations.

%Integrated and tested commercial O\&M servers for LTE system.

Contributing\marginnote{Contributions to Plan 9 port

\noindent \url{http://codereview.appspot.com/user/jdc}

\noindent \

\noindent Messages to 9fans

\noindent \url{http://9fans.net/archive/?q=from:jason.catena\&go=Grep}

\noindent \

\noindent Messages to golang-nuts

\noindent \url{http://j.mp/7eDOXx}}
to the open-source Plan 9 research operating system and Go programming
language.

Fluent in shell scripts, makefiles, and C with lex.  Learning Haskell,
Ruby, and Go. Familiar with yacc, SQL, C++, UML models, and Java.

Wrote in 2009 a presentation layer to graph and report key performance
indicator statistics, and improve collection within real-time modem
software of an LTE 4G base station and server gateway.

Improved build system and wrote tests for interface between subscriber
and base radio in 1998.  Implemented Van Jacobsen header compression in
the iDEN subscriber unit, and other improvements to the same, from 1994
to 1997.


\pagebreak

\subsection{Manage people and software}

My\marginnote{\url{http://swtools.wordpress.com/2009/04/02/my-experience-with-master-craftsman-teams/}}
experience with master-craftsman teams.

Integrated and reviewed ten years of contributions to two
multiple-product build systems, from dozens of team members and hundreds
of developers at a dozen sites world-wide.

Designed branching schemes for two dozen parallel branches in a
half-dozen concurrent releases for more than 500 engineers at a dozen
sites world-wide.

Controlled versions with ClearCase.  Tracked defects with Bugzilla,
ClearQuest, and DDTs.


\subsection{Communicate and organize}

Technical\marginnote{\url{http://swtools.wordpress.com/}

\noindent \url{http://www.evernote.com/pub/catena/public/}

\noindent \url{http://catenate.github.com/}}
solutions, writing, and interests.

Prototyped second major build system in \LaTeX\ with noweb to create a
series of literate programs.  These contain both code and documents in
one source file, organized for learning and maintenance, and extracted
to proper arrangement for execution.

Automatically collected and presented key performance indicators with
gnuplot as either real-time data, or a static graph with muted borders,
or as a set of sparklines.  Minimized marks on data points to observe
trends in data, as recommended by Edward Tufte.

%Adopted early, and promoted for work, social media tools as they became
%available on our intranet: Jabber instant messaging, Compass and Source
%Forge FAQs, Compass and Lyceum blogs, Compass wiki and twiki, Bugzilla
%defect tracking, and Laconica microblogging.

%Use social media to hone and share my professional experience.
Reputation\marginnote{\url{http://stackoverflow.com/users/27685/jason-catena}}
200 from 18 answers on Stack Overflow.
1386\marginnote{\url{http://delicious.com/swtools}} bookmarks on
programming, Haskell, Twitter, programming languages, design,
management, and programmers.

Tufte-influenced\marginnote{\url{https://dl.getdropbox.com/u/502901/twient.pdf}}
Twitter client design.


\subsection{Research and learn}

Prototyped a functional, literate, and aspect-oriented parallel build
system, to replace one developed over ten years of practice.
Studied academic papers on these techniques to implement new
methods of implementing and varying shell scripts and makefiles.

Learning\marginnote{Rewrote recursive and fold versions of Prelude list
functions.

\noindent\url{https://dl.getdropbox.com/u/502901/haskell.pdf}}
functional idioms by working through examples and exercises in
\emph{Real World Haskell}.  Updating C-family idioms by working through
Go tutorials.

Member of ACM since 2007.  Subscribe to ACM Digital Library.

Completed\marginnote{University of Illinois at Chicago

\noindent 1994--1998

\noindent \ 

\noindent Illinois Institute of Technology

\noindent 1995 and 1996}
most graduate work toward MS in Computer Science.  Studied advanced
topics in concurrent computing systems, distributed computing systems,
object-oriented programming languages and environments (C++ and
Smalltalk), user interface design, comparative computer architecture,
and concurrent programming.

Earned BS in Computer Engineering from The Pennsylvania State University
in 1994, and initiation to HKN Epsilon.

\end{document}
